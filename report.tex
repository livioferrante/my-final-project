\documentclass{article}
\begin{document}
PhD student Livio Ferrante
Effective Programming Practices - A short report on my hypothetical final project

|
I've created a private repository on Github in my educational account. The URL is https://github.com/livioferrante/my-final-project .
||
In my repository, I've cloned the template that you provided the last day of the course. I'll hypothetically use Python, otherwise I should switch on the branch to specify the language that I'll use most (e.g. stata, r...).
As I showed to you, Waf doesn't run in my machine, then I'm using my repository to put this file about what I would do... if Waf runs!! Therefore, I created a new branch called "report" and I'm working in it  ( git branch report   /   git checkout report ) with a tex file.
|
The ratio behind the use of Waf lies in the reproducibility. Waf is a tool, written in Python, that allows to automate the dependency tracking via a DAG structure (directed acyclic graphs). The replicability of results is an important issue in scientific research and is became a fundamental tool in many economic journals that are implementing strict replication policies. Moreover, it allows to avoid huge waste of time and resources and potential errors.
One could do all it manually by typing stuff to the console and clicking buttons. But it would be hard to track the sequence of clicks and runs one had to do to end up with the final document. So Waf is basically a long list of commands that runs everything in the right order to end up with the final document, presentation and several other things. To make it a little more convenient for the user, Waf can check different folders for the commands, so one doesn't need a huge wscript but can have several small ones located where they'd logically belong.
Currently, I haven't a personal project, then I imagined to implement a simple research which could lead to a creation of an empirical paper.

\end{document}